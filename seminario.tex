\documentclass{llncs}

\usepackage{makeidx}  

\begin{document}

\frontmatter          % for the preliminaries
%

%
\chapter*{Preface}
%
En este documento se intenta abordar sobre la recucuperación de información basada en imágenes.


\tableofcontents
%
\mainmatter              % start of the contributions
%
\title{Principales m\'etodos de Image Information Retrieval}
%
\titlerunning{}  % abbreviated title (for running head)
%                                     also used for the TOC unless
%                                     \toctitle is used
%
\author{Ivar Ekeland\inst{1} \and Roger Temam\inst{2}
Jeffrey Dean \and David Grove \and Craig Chambers \and Kim~B.~Bruce \and
Elsa Bertino}

%
\author{David Orlando De Quesada Oliva, Javier Dom\'inguez}
%
\authorrunning{Ivar Ekeland et al.} % abbreviated author list (for running head)
%
%%%% list of authors for the TOC (use if author list has to be modified)
\tocauthor{Ivar Ekeland, Roger Temam, Jeffrey Dean, David Grove,
Craig Chambers, Kim B. Bruce, and Elisa Bertino}
%
\institute{MATCOM, Universidad de La Habana,\\
\email{d.quesada2@estudiantes.matcom.uh.cu, j.dominguez@estudiantes.matcom.uh.cu},\\
\texttt{}
}
\begin{abstract}
    \dots
   \keywords{}
   \end{abstract}


\maketitle

\section{Keyword Based Image retrieval:}

En un \textbf{image retrieval system convencional}, los \textbf{keywords} son 
usados como descriptores para indexar y recuperar una imagen.  Las palabras 
claves (key words) no transmiten mejor que el contenido de una imagen 
el significado de esta. Antes de que las imágenes sean almacenadas en la base de datos, son examinadas 
manualmente y se les asigna una palabra  clave (\textbf{keyword}) para describir su 
contenido. Estos  keywords  son almacenados como parte de los atributos 
asociados a la imagen. En el proceso de hacer una consulta, el sistema 
aceptará del usuario una o varias \textbf{keywords} que serán el criterio de búsqueda.
Luego se realiza un proceso para encontrar las imágenes que cumplen con el criterio 
de búsqueda. Las t\'ecnicas de text based image retrieval usa texto para describir el 
contenido de una imagen lo que a menudo crea ambig$\ddot{u}$edad e insuficiencia en el 
procesamiento de la query y el rendimiento en una b\'usqueda de la base de datos de 
im\'agenes. El proceso de asignaci\'on de meta datos con captions o keywords a una imagen
digital es conocido como anotaci\'on autom\'atica de im\'agenes(automatic image anotation).
Este tipo de text based informacion retrieval est\'a motivado l\'exicamente en lugar de 
conceptualmente, lo que lleva a resultados de b\'usqueda irrelevantes en la recuperaci\'on en la 
recuperaci\'on de informaci\'on.

\subsection{Text Based Image Retrieval:}

Las t\'ecnicas b\'asicas de recuperaci\'on de documentos pueden ser usadas para la recuperaci\'on
de im\'agenes basadas en metadatos sin modificaci\'on. 
\newpage

\end{document}