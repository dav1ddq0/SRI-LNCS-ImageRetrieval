\documentclass{llncs}

\usepackage{makeidx}  

\begin{document}

\frontmatter          % for the preliminaries
%


\tableofcontents
%
\mainmatter              % start of the contributions
%
\title{Principales m\'etodos de Image Information Retrieval}
%
\titlerunning{}  % abbreviated title (for running head)
%                                     also used for the TOC unless
%                                     \toctitle is used
%

%
\author{David Orlando De Quesada Oliva, Javier Dom\'inguez}
%
\authorrunning{Ivar Ekeland et al.} % abbreviated author list (for running head)
%
%%%% list of authors for the TOC (use if author list has to be modified)

%
\institute{MATCOM, Universidad de La Habana,\\
\email{d.quesada2@estudiantes.matcom.uh.cu, j.dominguez@estudiantes.matcom.uh.cu},\\
\texttt{}
}
\begin{abstract}
    \dots
   \keywords{}
   \end{abstract}


\maketitle

\section{Caracter\'isticas del problema:}

La Recuperación de im\'agenes es el campo que se encarga de buscar y obtener imágenes 
digitales de una base  de datos. Debido a la cantidad creciente de imágenes digitales 
alrededor del mundo, desde 1970 este campo ha estado bien activo. Un sistema de 
recuperación efectivo y rápido de imágenes necesita operar sobre una colección de 
imágenes y devolver las imágenes relevantes basadas en la consulta, la cual 
se realiza lo más cercana posible a la percepción humana. Los investigadores de este 
campo poco a poco, han ido mejorando e implementando varios tipos de sistemas de recuperación 
de imágenes, de los sistemas basados en \textbf{palabras claves}, pasando por los sistemas 
basados en el contenido (características) de una imagen, y finalmente llegando a los 
sistemas de recuperación semánticos, con el objetivo de reducir el vacío semántico que 
existe entre la representación de características de bajo nivel (color, textura, forma, etc) y 
la semántica de alto nivel en las imágenes.

Desarrollar un motor de búsqueda de imágenes omnipotente, capaz de satisfacer a todos los usuarios 
requiere entender y caracterizar la interacción y la búsqueda de imágenes desde el punto de vista 
del usuario y del sistema. Desde el punto de vista del usuario, claridad en lo que desea, donde lo 
quiere buscar y de que forma quiere realizar su consulta. Desde el punto de viste del motor de 
búsqueda, como desea el usuario que le sean presentados los resultados de su consulta, dónde desea 
buscar el usuario y cual es la naturaleza de la consulta del usuario.

\section{Keyword Based Image retrieval:}

En un \textbf{image retrieval system convencional}, los \textbf{keywords} son 
usados como descriptores para indexar y recuperar una imagen.  Las palabras 
claves (key words) no transmiten mejor que el contenido de una imagen 
el significado de esta. Antes de que las imágenes sean almacenadas en la base de datos, son examinadas 
manualmente y se les asigna una palabra  clave (\textbf{keyword}) para describir su 
contenido. Estos  keywords  son almacenados como parte de los atributos 
asociados a la imagen. En el proceso de hacer una consulta, el sistema 
aceptará del usuario una o varias \textbf{keywords} que serán el criterio de búsqueda.
Luego se realiza un proceso para encontrar las imágenes que cumplen con el criterio 
de búsqueda. Las t\'ecnicas de text based image retrieval usa texto para describir el 
contenido de una imagen lo que a menudo crea ambig$\ddot{u}$edad e insuficiencia en el 
procesamiento de la query y el rendimiento en una b\'usqueda de la base de datos de 
im\'agenes. El proceso de asignaci\'on de meta datos con captions o keywords a una imagen
digital es conocido como anotaci\'on autom\'atica de im\'agenes(automatic image anotation).
Este tipo de text based informacion retrieval est\'a motivado l\'exicamente en lugar de 
conceptualmente, lo que lleva a resultados de b\'usqueda irrelevantes en la recuperaci\'on en la 
recuperaci\'on de informaci\'on.

\subsection{Text Based Image Retrieval:}

Las t\'ecnicas b\'asicas de recuperaci\'on de documentos pueden ser usadas para la recuperaci\'on
de im\'agenes basadas en metadatos sin modificaci\'on. En un keyword based image
retrieval, los metadatos que describen las im\'agenes pueden ser categorizados en 2
partes. Una parte se refiere a las herramientas usadas en el proceso de creaci\'on de la imagen, estilo de
de arte de la imagen, artista, precio, y otras propiedades expl\'icitas de la imagen. La otra parte describe
lo que realmente hay en la imagen, las propiedades imp\'icitas que pueden entenderse al 
percibir la imagen en si. En el contexto actual de la recuperaci\'on, el texto plano anotado en im\'agenes
responde de manera similar al texto plano en documentos, debido a que ambos contienen texto, lo cual permite
que sean explotados por las t\'ecnicas convencionales de text-bases information retrieval. La recuperaci\'on de
informaci\'on basada en texto gen\'erica se realiza de tal manera que inicialmente el usuario reliza una
consulta(query) que tiene de 1 a m \textbf{keywords}. En los sistemas de recuperaci\'on basados en 
metadatos(metadata based information retrieval), el buscador compara los keywords con un conjunto de im\'agenes 
recopiladas de una base de datos y les da prioriad a los valores. Por ejemplo, si el keyword es \textbf{book}, y la 
imagen A contiene 2 ocurrencias de book y la imagen B solo una ocurrencia, entonces A tiene una prioridad mayor.  Las 
im\'agenes con palabras claves anotadas son mostradas al usuarios en el orden de reducci\'on de la prioridad. Im\'agenes 
irrelevantes son recuperadas y el usuario tiene que gastar tiempo en el filtrado de la informaci\'on, usualmente navegando
a trav\'es de los resultados de b\'usqueda. 

\subsection{Field Based Image Retrieval:}
Field based retrieval es una extensi\'on del text based retrieval donde solo un campo(field) es usado en anotaci\'on 
y recuperaci\'on. El enfoque basado en el campo (field based) describe y recupera art\'iculos usando uno o m\'as pares de valores 
del campo. Regularmente un esquema de metadatos es descrito por un conjunto de campos y pocas indicaciones sobre el tipo de
valores que puede ser elegidos por un campo  particular. La plantilla (template) de metadatos y esquemas ampliamaneta utilizada para 
describir documentos online en general es la \textbf{Dublin Core(DC)}. Los campos de la DC version 1.1 son rights, coverage, relation,
language, source, identifier, format, type, data, contributor, publisher, description, subject, creator y title.  Versiones calificadas de 
DC han side creadas para dominions particulares como la decripci\'on de piezas de arte en museos.

\subsection{Structure Based Image Retrieval:}

El paradigma de recuperaci\'on basado en estructuras. En este m\'etodo, se utiliza un enfoque basado enn el campo(field) que principalmente
utiliza una estructura de parers de valores atributo. Este m\'etodo permite descripciones m\'as complejas implicando relaciones. Por ejemplo,
una definici\'on de una parte de un auto puede incluir especificaciones de esos componentes. Cada elemento del objeto se puede especificar 
de nuevo usando varios atributos como la forma, el tama\~{n}o y el material. Los elementos pueden incluso tener elementos ellos mismos, por 
ejemplo, una mesa tiene patas, y sus subelementos pueden moverse hasta el nivel donde un elemento no puede obtener un subelemento m\'as particular.







\newpage

\end{document}